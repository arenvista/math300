\documentclass[11pt]{article}
\usepackage{amsmath,amsthm,amscd,amssymb,mathrsfs,centernot}
\usepackage{latexsym,epsf,epsfig}
\newcommand{\ds}{\displaystyle}
\newcommand{\sU}{\mathscr U}
\usepackage[left=1.2in, right=1.2in, top=1in, bottom=1in]{geometry}
%The above set the parameters adjust the margins. There are many ways to do this, and frankly, what is above is not the best. However, it will work the time being.
\usepackage{sectsty}


\usepackage{booktabs, multirow} % for borders and merged ranges
\usepackage{soul}% for underlines
\usepackage{xcolor,colortbl} % for cell colors
\usepackage{changepage,threeparttable} % for wide tables

\usepackage{hyperref}
\theoremstyle{plain}
\newtheorem{theorem}{Theorem}

\usepackage{graphicx}

% Margins
\topmargin=-0.45in
\evensidemargin=0in
\oddsidemargin=0in
\textwidth=6.5in
\textheight=9.0in
\headsep=0.25in

\title{ MATH 300: Homework 3}
\author{ Aren Vista }
\date{\today}

\begin{document}
\maketitle	
\pagebreak

\section{14.3 a,c}

For each of the following relations defined on the set {1,2,3,4,5} determine whether the relation is reflexive, irreflexive, symmetric, antisymetric, and or transitive.

\begin{center}
Recall the following properties:
\end{center}

\begin{itemize}
    \item $R$ is called a reflexive iff $\forall ~a \in A, ~aRa$
    \item $R$ is called a irrreflexive iff $\forall ~a \in A, (a,a) \not\in R$
    \item $R$ is symmetric iff $aRb \implies bRa$ when $a,b \in A$
    \item $R$ is antisymetric iff $aRb \land bRa \implies a=b$ 
    \item $R$ is transitive iff $aRb \land bRc \implies aRc$ 
\end{itemize}

$$R = \{(1,1), (2,2), (3,3), (4,4), (5,5)\}$$
\begin{itemize}
    \item $aRa ~\forall ~a \in A$ therefore is reflexive
    \item $\forall (a,b) \in A, \exists (b,a) \in A \therefore aRb \implies bRa$ therefore is symmetric
    \item $a=1, (1,1) \in R \implies (a,a) \in R \therefore$ not irreflexive
    \item $R$ is antisymetric iff $aRb \land bRa \implies a=b$ Observe $aRb \land bRa$ is false; therefore is vaccously true;
    \item $R$ is transitive iff $aRb \land bRc \implies aRc$. Observe $aRb \land bRc$ is false; therefore is vaccously true;
\end{itemize}


$$R = \{(1,1), (1,2), (1,3), (1,4), (1,5)\}$$
\begin{itemize}
    \item $\exists a=2 \ni (2,2) \not\in R \therefore$ R is not reflexive
    \item $a=1, (1,1) \in R \implies (a,a) \in R \therefore$ not irreflexive
    \item Obs $\exists ~a=1, b=2 \ni aRb \centernot\implies bRa$ therefore is not symmetric
    \item $R$ is antisymetric iff $aRb \land bRa \implies a=b$ Observe $aRb \land bRa$ is false; therefore is vaccously true;
    \item $R$ is transitive iff $aRb \land bRc \implies aRc$. Observe $aRb \land bRc$ is false; therefore is vaccously true;
\end{itemize}


\section{14.15}
Prove: A relation on R on a set A is antisymetric iff 
$$R \cap R^{-1} \subset \{(a,a): a \in A\}$$
$$R \text{ is antisymetric iff } aRb \land bRa \implies a=b \therefore A = \{(a,b) | aRb \land bRa \implies a=b\}$$

Suppose \( R \) is antisymmetric.
\begin{itemize}
    \item If  $(a, b) \in R \cap R^{-1} $, then $ (a, b) \in R $ and $ (b, a) \in R^{-1} $ 
    \item By antisymmetry, $ a = b $
    \item Hence, $ (a, b) $ is of the form $ (a, a) $
    \item implying $ R \cap R^{-1} \subset \{(a,a): a \in A\} $
\end{itemize}


Suppose \( R \) is antisymmetric.
\begin{itemize}
    \item If  $(b, a) \in R \cap R^{-1} $, then $ (b, a) \in R $ and $ (a, b) \in R^{-1} $ 
    \item By antisymmetry, $ a = b $
    \item Hence, $ (b, a) $ is of the form $ (b, b) $
    \item implying $ R \cap R^{-1} \subset \{(b,b): a \in A\} $
\end{itemize}

Therefore \( R \) is antisymmetric iff \( R \cap R^{-1} \subset \{(a,a): a \in A\} \).

\section{14.16}
Give an example of a relation on a set that is both symmetric and transitive but not reflexive. 

$$R  = {(a,b),(b,a),(b,c),(c,b),(a,c),(c,a)}$$

Explain what is wrong with the "proof":

\begin{itemize}
    \item "Suppose R is symmetric and transitive"
    \item Symmetric means that $xRy \implies yRx$
    \item Applying transitivity to $xRy \land yRx$ to give $xRx$ therefore is reflexive
\end{itemize}

Answer:
\begin{itemize}
    \item "Suppose R is symmetric and transitive"
    \item Symmetric means that $\{(x,y) | xRy \implies yRx\}$ 
    \item Transitive $\{(x,y), (y,z), (x,z) | xRy \land yRz \implies xRz\}$
    \item Observe $(x,x) \not\in R \therefore$ not reflexive
\end{itemize}

\section{15.3a}
Which of the following are equivalence relations?
$$R = \{(1,1),(1,2),(2,1),(2,2),(3,3)\}, ~A = \{1,2,3\}$$
    $$\forall a \in R, ~aRa \therefore \text{ reflexive} \implies \forall a \in R, ~\exists (a,a) \in R $$
    $$\forall a,b \in R, ~aRb \implies bRa \therefore \text{ symmetric} \implies \forall a,b \in R, ~\exists (a,b),(b,a) \in R$$
    $$\forall a,b,c \in R, ~aRb \land bRc \implies aRc \therefore \text{ transitive by vaccuous truth}$$
Since R is reflexive, transitive, and symmetric, R is an equivalence relation

\section{15.7f}
For each equivalence relation below, find the requested equivalence class. 
\begin{itemize}
    \item $R$ is has the same size as A, on A=$2^{\{1,2,3,4,5\}}$. Find $[\{1,3\}]$
    $$ [\{1,3\}] = 
    \left\{\begin{array}{cccccc}
    \{1,1\}, & \{1,2\}, & \{1,3\}, & \{1,4\}, & \{1,5\}, \\
    \{2,2\}, & \{2,3\}, & \{2,4\}, & \{2,5\}, \\
    \{3,3\}, & \{3,4\}, & \{3,5\}, \\
    \{4,4\}, & \{4,5\}, \\
    \{5,5\} \\
    \end{array}\right\}
    $$
\end{itemize}

\end{document}
\begin{itemize}
    \item  
    \item 
    \item 
    \item 
    \item 
\end{itemize}
