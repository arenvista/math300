\documentclass[11pt]{article}
\usepackage{amsmath,amsthm,amscd,amssymb,mathrsfs}
\usepackage{latexsym,epsf,epsfig}
\newcommand{\ds}{\displaystyle}
\newcommand{\sU}{\mathscr U}
\usepackage[left=1.2in, right=1.2in, top=1in, bottom=1in]{geometry}
%The above set the parameters adjust the margins. There are many ways to do this, and frankly, what is above is not the best. However, it will work the time being.
\usepackage{sectsty}


\usepackage{booktabs, multirow} % for borders and merged ranges
\usepackage{soul}% for underlines
\usepackage{xcolor,colortbl} % for cell colors
\usepackage{changepage,threeparttable} % for wide tables

\usepackage{hyperref}
\theoremstyle{plain}
\newtheorem{theorem}{Theorem}

\usepackage{graphicx}

% Margins
\topmargin=-0.45in
\evensidemargin=0in
\oddsidemargin=0in
\textwidth=6.5in
\textheight=9.0in
\headsep=0.25in

\title{ MATH 300: Homework 5}
\author{ Aren Vista }
\date{\today}

\begin{document}
\maketitle	
\pagebreak

\section{Question 1}
Suppose $A,B,C$ are pairwise dijoint sets. Prove or disprove $| A \cup B \cup C| = |A| + |B| + |C|$

\begin{proof}
    Suppose $A, B, C$ are pairwise disjoint sets. Then:
    \begin{align*}
        |A \cap B| &= 0, \\
        |A \cap C| &= 0, \\
        |B \cap C| &= 0, \\
        |A \cap B \cap C| &= 0.
    \end{align*}
    By the principle of inclusion-exclusion, we have:
    \[
    |A \cup B \cup C| = |A| + |B| + |C| - |A \cap B| - |A \cap C| - |B \cap C| + |A \cap B \cap C|.
    \]
    Substituting the values from above, we get:
    \[
    |A \cup B \cup C| = |A| + |B| + |C| - 0 - 0 - 0 + 0.
    \]
    Therefore, $|A \cup B \cup C| = |A| + |B| + |C|$.
\end{proof}

\section{Question 2}
Let $A,B$ be nonempty sets. Prove $A \times B = B \times A$ iff $A=B$

\begin{proof}
    Suppose $A = B$. \\
    Observe $A \times B = A \times A = B \times B = B \times A$ by definition of $A = B$. \\ \\
    Suppose $A \times B = B \times A$.\\
    Observe $A \times B = \{(a,b) | a\in A, b \in B\} = C$\\
    Observe $B \times A = \{(b,a) | a\in A, b \in B\} = D$\\
    Then $A \times B = B \times A \equiv \{(a,b) | a\in A, b \in B\} =\{(b,a) | a\in A, b \in B\}$\\ \\
    Observe $\exists ~b_0 \in B \ni \forall a \in A, (a,b_0) \in A \times B$ \\
    By supposition $(a,b_0) \in B \times A$\\
    By defn. of $B \times A \implies a \in B \land b_0 \in A$ \\
    $\therefore A \subset B$\\ \\
    Observe $\exists ~a_0 \in A \ni \forall b \in B, (a_0,b) \in A \times B$ \\
    By supposition $(a_0,b) \in B \times A$\\
    By defn. of $B \times A \implies a_0 \in B \land b \in A$ \\
    $\therefore B \subset A$\\ \\
    $A \subset B \land B \subset A \implies A=B$\\
    Thus both directions of the biconditional have been proven and we can conclude. $A \times B = B \times A$ iff $A=B$
\end{proof}

\section{Question 3}
For each of the following statements, determine whether it is true or false. If the statement is true, provide a proof; if it is false, present a counterexample.

$$|A-B| = |A| - |B|$$
\begin{proof}
    \item Suppose $B = \{b_1, b_2\}$ and $A = \{a_1, b_1\}$
    \item Observe $A-B = \{a_1\}$
    \item Observe $|A-B| = 1$ and $|A| - |B| = 2 - 2 = 0$
    \item $1 \neq 0 \therefore |A-B| \neq |A| - |B|$
    \item $|A-B| = |A| - |B|$ is false by counterexample
\end{proof}

\section{Question 4}
Let $X = \{1,2,3\}.$ For two subsets $A,B \in X$, define a relation $ARB$ if $|A| = |B|$. Show this is an equivalence relation. What is the equivalence class contianing $\{1\}$\\
Suppose $A,B \in X \implies a \in A \land a \in X, b \in B \land b \in X$\\
$ARB = \{(A,B) : |A| = |B|\}$ \\
$|A| = |B| \iff |B| = |A| \implies ARB \land BRA$ therefore symmetric\\
$ARB \land BRC \implies |A| = |B| \land |B| = |C| \implies |A| = |C| \therefore ARC \in R$ therefore is transitive\\
$\forall A, |A| = |A| \implies ARA \in R$ therefore is reflexive\\
Since is symmetric, transitive, and reflexive, R is an equivalence relation.
$$[\{1\}]=\{\{1\}, \{2\}, \{3\}\} $$

\section{Question 5}
Given two sets $A$ and $B$ with $|A|=3$ and $|B|=4$ what is the cardinality of the power set of $A \times B$\\
$|A \times B| = |A| \times |B|$ = 3*4=12\\
$P|A \times B| = 2^{|A \times B|} = 2^{12}$
\end{document}
