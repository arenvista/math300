\documentclass[11pt]{article}
\usepackage{amsmath,amsthm,amscd,amssymb,mathrsfs}
\usepackage{latexsym,epsf,epsfig}
\newcommand{\ds}{\displaystyle}
\newcommand{\sU}{\mathscr U}
\usepackage[left=1.2in, right=1.2in, top=1in, bottom=1in]{geometry}
%The above set the parameters adjust the margins. There are many ways to do this, and frankly, what is above is not the best. However, it will work the time being.
\usepackage{sectsty}


\usepackage{booktabs, multirow} % for borders and merged ranges
\usepackage{soul}% for underlines
\usepackage{xcolor,colortbl} % for cell colors
\usepackage{changepage,threeparttable} % for wide tables

\usepackage{hyperref}
\theoremstyle{plain}
\newtheorem{theorem}{Theorem}

\usepackage{graphicx}

% Margins
\topmargin=-0.45in
\evensidemargin=0in
\oddsidemargin=0in
\textwidth=6.5in
\textheight=9.0in
\headsep=0.25in

\title{ MATH 300: Homework 5}
\author{ Aren Vista }
\date{\today}

\begin{document}
\maketitle	
\pagebreak

\section{Question 1}

Show $\forall n \in \mathbb{N}, 1^3+2^3+...+n^3 = \Biggr[ \dfrac{n(n+1)}{2} \Biggl] ^2$

\begin{itemize}
    \item Let $P(n): 1^3+2^3+...+n^3 = \Biggr[ \dfrac{n(n+1)}{2} \Biggl] ^2$
    \item Observe $P(1): 1^3 = \Biggr[ \dfrac{(1)((1)+1)}{2} \Biggl] ^2 = 1$ therefore $P(1)$ is true
    \item Suppose $k \in \mathbb{N}, P(k): 1^3+2^3+...+k^3 = \Biggr[ \dfrac{k(k+1)}{2} \Biggl] ^2$
    \item Observe $\Biggr[ \dfrac{k(k+1)}{2} \Biggl] ^2 = \Biggr[ \dfrac{k^4+2k^3+k^2}{4} \Biggl]$
    \item Observe $\Biggr[ \dfrac{(k+1)^4+2(k+1)^3+(k+1)^2}{2} \Biggl] = \Biggr[ \dfrac{k^4+6k^3+13k^2+12k+4}{4} \Biggl]$
    \item Add $(k+1)^3$ to both sides of $1^3+2^3+...+k^3 = \Biggr[ \dfrac{k(k+1)}{4} \Biggl] ^2$
    \item This yields $1^3+2^3+...+k^3 + (k+1)^3 = \Biggr[ \dfrac{k(k+1)}{2} \Biggl] ^2 + (k+1)^3$
    \item Observe $(k+1)^3 = k^3+3k^2+3k+1 = \dfrac{4k^3+12k^2+12k+4}{4}$
    \item Subsituting $1^3+2^3+...+k^3 + (k+1)^3 = \Biggr[ \dfrac{k(k+1)}{2} \Biggl] ^2 + \dfrac{4k^3+12k^2+12k+4}{4}$
    \item Subsituting $1^3+2^3+...+k^3 + (k+1)^3 = \Biggr[ \dfrac{k^4+2k^3+k^2}{4} \Biggl] + \dfrac{4k^3+12k^2+12k+4}{4}$
    \item Simplify $1^3+2^3+...+k^3 + (k+1)^3 = \Biggr[ \dfrac{k^4+6k^3+13k^2+12k+4}{4} \Biggl]$
    \item Therefore $P(k+1)$ is true
    \item By PMI $P(n)$ is true $\forall n \in \mathbb{N}$
\end{itemize}

\section{Question 2}
Show $\forall n \in \mathbb{N}, 1+4+7+...+(3n-2) = \dfrac{3n^2-n}{2}$

\begin{itemize}
    \item Let $P(n): 1+4+7+...+(3n-2) = \dfrac{3n^2-n}{2}$
    \item Observe $P(1): 1 = \dfrac{3(1)^2-(1)}{2} = 1$ therefore $P(1)$ is true
    \item Suppose $k \in \mathbb{N}, P(k): 1+4+7+...+(3k-2) = \dfrac{3k^2-k}{2}$
    \item Add $(3k+1)$ to both sides of $1+4+7+...+(3k-2) = \dfrac{3k^2-k}{2}$
    \item Thus $1+4+7+...+(3k-2) + (3k+1) = \dfrac{3k^2-k}{2} + (3k+1)$
    \item Observe $\dfrac{3(k+1)^2-(k+1)}{2} = \dfrac{3k^2+5k+2}{2}$
    \item Simplify $1+4+7+...+(3k-2) + (3k+1) = \dfrac{3k^2-k}{2} + \dfrac{(6k+2)}{2}$
    \item Observe $1+4+7+...+(3k-2) + (3k+1) = \dfrac{3k^2+5k+2}{2}$
    \item Therefore $P(k+1)$ is true
    \item By PMI $P(n)$ is true $\forall n \in \mathbb{N}$
\end{itemize}

\section{Question 3}
Show $\forall n \in \mathbb{N}, n<2^n$

\begin{itemize}
    \item Let $P(n):n<2^n$
    \item Observe $P(1):1 < 2^1$ thus, $P(1)$ is true
    \item Suppose $k \in \mathbb{N}, P(k): k<2^k$
    \item Add $1$ to both sides of $k<2^k$ to get $k+1<2^k+1$
    \item Observe $2^{k+1} = 2^k*2$
    \item Clearly $2^k+1 < 2^{k+1}$
    \item Observe $k+1 < 2^k+1 < 2^{k+1}$
    \item Therefore $P(k+1)$ is true
    \item By PMI $P(n)$ is true $\forall n \in \mathbb{N}$
\end{itemize}

\section{Question 4}
Show $\forall n \in \mathbb{N}, 6|(n^3+5n)$
\begin{itemize}
    \item Let $P(n): 6|(n^3+5n)$
    \item Let $P(1): 6|(1^3+5n) \equiv 6|(1+5) \equiv 6|6$ therefore $P(1)$ is true
    \item Suppose $k \in \mathbb{N}, P(k):(k^3+5k) \implies \exists c \in \mathbb{N} \ni 6c=k^3+5k$
    \item Observe $(k+1)^3 + 5(k+1) = k^3+3k^2+8k+5$
    \item Add $3k^2 + 3k + 6$ to both sides of $6c=k^3+5k$
    \item Thus $6c+3k^2+3k+6=k^3+3k^2+8k+6$
    \item Group $6c+(3k^2+3k)+6=k^3+3k^2+8k+6$
    \item Simplify $6c+3k(k+1)+6=(k+1)^3 + 5(k+1)$
    \item Let $m \in \mathbb{N} \ni k=2m \implies 6m(2m+1) = 12m + 6$ therefore is divsible by two
    \item Let $m \in \mathbb{N} \ni k=2m+1 \implies 6(2m+1)(2m+2) = (12m+6)(2m+2) = 24m^2+24m+12m+12$ therefore is divsible by two
    \item Let $n \in \mathbb{N} \ni k=2n$ Subsitute $6c+3k(2n)+6=(k+1)^3 + 5(k+1)$
    \item Simplify $6c+6kn+6=(k+1)^3 + 5(k+1)$
    \item Simplify $6(c+kn+1)=(k+1)^3 + 5(k+1)$
    \item Let $(c+kn+1)=z, z \in \mathbb{N}$
    \item Subsitute $6z=(k+1)^3 + 5(k+1)$
    \item Therefore $P(k+1): 6|(k+1)^3 + 5(k+1)$ is true.
    \item By PMI $P(n)$ is true $\forall n \in \mathbb{N}$
\end{itemize}


\end{document}

