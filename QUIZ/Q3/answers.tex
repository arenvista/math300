\documentclass[11pt]{article}
\usepackage{amsmath,amsthm,amscd,amssymb,mathrsfs,tikz}
\usepackage{latexsym,epsf,epsfig}
\newcommand{\ds}{\displaystyle}
\newcommand{\sU}{\mathscr U}
\usepackage[left=1.2in, right=1.2in, top=1in, bottom=1in]{geometry}
%The above set the parameters adjust the margins. There are many ways to do this, and frankly, what is above is not the best. However, it will work the time being.
\usepackage{sectsty}


\usepackage{booktabs, multirow} % for borders and merged ranges
\usepackage{soul}% for underlines
\usepackage{xcolor,colortbl} % for cell colors
\usepackage{changepage,threeparttable} % for wide tables

\usepackage{hyperref}
\theoremstyle{plain}
\newtheorem{theorem}{Theorem}

\usepackage{graphicx}

% Margins
\topmargin=-0.45in
\evensidemargin=0in
\oddsidemargin=0in
\textwidth=6.5in
\textheight=9.0in
\headsep=0.25in

\title{ MATH 300: Quiz 3 Answers}
\author{ Aren Vista }
\date{\today}

\begin{document}
\maketitle	
\pagebreak

\section*{Supposition}
Let $f: X \rightarrow Y$ be a function, $A \subset X$ and $C \subset Y$ Then:

\[
    f(A):=\{f(a)|a\in A \}
\]

\[
    f^{-1}(C) = \{x\in X | f(x) \in C\}
\]
\subsection*{P1}
Show: $f: X \rightarrow Y$ is a function $\implies ~\forall$ sets $A \subset X, A \subset f^{-1}(f(A))$ 

Observe
$f: X \rightarrow Y$ is a function $\implies \forall x \in X, f(x) \in Y$

From supposition
\[
    f^{-1}(C) = \{x\in X | f(x) \in C\}
\]

Recall $A \subset X \implies \forall ~a \in A, a \in X$

Thus, 
$f: A \rightarrow Y$ is a function $\implies \forall a \in X, f(a) \in Y$ must hold

Therefore
\[
    B = \{a \in A | f(a) \in C\}, B  \subset f^{-1}(C) 
\]

Thus \[
    \forall a \in A, a \in B
\]
\subsection*{P1*}
Let $a$ be an arbitrary element in $A$

By the defn of the image set \[
    f(a) \in f(A)
\]

As $A \subset X, f(X) \in C$ \[
    f(A) \subset C
\]


By the defn of the preimage \[
    f^{-1}(C)=\{
        x \in X | f(x) \in C
    \}
\]

By subsititon the preimage of must also hold for \[
    f^{-1}(f(A))=\{
        x \in X | f(x) \in f(A)
    \}
\]

Since $f(a) \in f(A)$: \[
    a \in f^{-1}f(A)
\]

\subsection*{P2}
Show: If $f: X \rightarrow Y$ is injective, then for any two sets $A$ and $B$ in $X$\[
    f(A \cap B) = f(A) \cap f(B)
\]

Case 1: Prove $f(A \cap B) \subset f(A) \cap f(B)$

Let \[
    y \in f(A \cap B)
\]

Thus \[
    \exists ~x \in A \cap B \ni f(x) = y
\]

By defn \[
    x \in A \cap B \implies x \in A \land x \in B
\]

Since $x \in A \land f(x) = y$\[
    y \in f(A) 
\]

Since $x \in B \land f(x) = y$\[
    y \in f(B) 
\]

Since $y \in f(B) \land y \in f(A)$\[
    y \in f(A) \land f(B)
\]

Thus we can conclude:\[
    (A \cap B) \subset f(A) \cap f(B)
\]

Case 2: Prove $f(A) \cap f(B) \subset (A \cap B)$

Let \[
    y \in f(A) \cap f(B)
\]

Then \[
    y \in f(A) \land y \in f(B)
\]

Since \[
    y \in f(A) \implies \exists ~a \in A \ni f(a) = y
\]

Since \[
    y \in f(B) \implies \exists ~b \in B \ni f(b) = y
\]

Naturally \[
    f(a) = f(b) = y
\]

Since $f$ is injective \[
    f(a) = f(b) \implies a=b
\]

Let \[
    x = a = b
\]

Since \[
    x \in A \land x \in B, x \in A \cap B 
\]

Thus we can conclude:\[
    f(A) \cap f(B) \subset (A \cap B)
\]

By Case 1 and Case 2  we conclude:\[
    f(A) \cap f(B) = (A \cap B)
\]

\subsection*{P3}
$A$ is a set with 5 elements. If $f:A \rightarrow A$ is a function, what are the minimum and maximum values of $|f(A)|$. 
When is the maximum achieved?

Let\[
    A = \{a_1,a_2,a_3,a_4,a_5\}
\]

If $f: A \rightarrow A$ is a function\[
    f(a) = a_x \land f(a) = a_y \implies a_x = a_y
\]

Thus at most:\[
    f(A) = \{ 
        a_1,a_2,a_3,a_4,a_5
    \}
\]

Thus at least\[
    f(A) = \{ 
        a_1
    \}
\]

Meaning the min cardinality is 1, and the max is 5.

\subsection*{P4}
Show the function $f$ is bijective:\[
    f:(0,\inf) \rightarrow (0,1), 
\]

\[
    f(x) = \frac{x^2}{1+x^2}
\]

$f$ is bijective iff $f$ is injective and $f$ is surjective

Prove injectivity:

Let $a,b \in (0,\inf)$
\[
    \frac{a^2}{1+a^2} = 
    \frac{b^2}{1+b^2}
\]

\[
    a^2(1+b^2) = 
    b^2(1+a^2)
\]

\[
    a^2+a^2b^2 = 
    b^2+a^2b^2
\]

\[
    a^2 = b^2
\]

\[
    a = b
\]

Thus $f$ is injective

Prove surjectivity:

Let $a \in (0,\inf), b \in (0,1)$

\[
    \frac{a^2}{1+a^2} = 
    b
\]

\[
    a^2 = 
    b(1+a^2)
\]

\[
    a^2 = 
    b+ba^2
\]

\[
    a^2-ba^2 = 
    b
\]

\[
    a^2(1-b) = 
    b
\]

\[
    a^2 = 
    \frac{b}{(1-b)}
\]

\[
    a = 
    \sqrt{\frac{b}{(1-b)}}
\]

Thus $f$ is surjective. 

As $f$ is both injective and surjective. It must be bijective as defn.

\end{document}
